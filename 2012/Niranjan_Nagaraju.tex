%%%%%%%%%%%%%%%%%%%%%%%%%%%%%%%%%%%%%%%%%%%%%%%%%%%%%%%%%%%%%%%%%%%%%%%%
%%%%%%%%%%%%%%%%%%%%%% Simple LaTeX CV Template %%%%%%%%%%%%%%%%%%%%%%%%
%%%%%%%%%%%%%%%%%%%%%%%%%%%%%%%%%%%%%%%%%%%%%%%%%%%%%%%%%%%%%%%%%%%%%%%%

%%%%%%%%%%%%%%%%%%%%%%%%%%%%%%%%%%%%%%%%%%%%%%%%%%%%%%%%%%%%%%%%%%%%%%%%
%% NOTE: If you find that it says                                     %%
%%                                                                    %%
%%                           1 of ??                                  %%
%%                                                                    %%
%% at the bottom of your first page, this means that the AUX file     %%
%% was not available when you ran LaTeX on this source. Simply RERUN  %%
%% LaTeX to get the ``??'' replaced with the number of the last page  %%
%% of the document. The AUX file will be generated on the first run   %%
%% of LaTeX and used on the second run to fill in all of the          %%
%% references.                                                        %%
%%%%%%%%%%%%%%%%%%%%%%%%%%%%%%%%%%%%%%%%%%%%%%%%%%%%%%%%%%%%%%%%%%%%%%%%

%%%%%%%%%%%%%%%%%%%%%%%%%%%% Document Setup %%%%%%%%%%%%%%%%%%%%%%%%%%%%

% Don't like 10pt? Try 11pt or 12pt
\documentclass[9pt]{article}

% This is a helpful package that puts math inside length specifications
\usepackage{calc}

% Simpler bibsection for CV sections
% (thanks to natbib for inspiration)
\makeatletter
\newlength{\bibhang}
\setlength{\bibhang}{1em}
\newlength{\bibsep}
 {\@listi \global\bibsep\itemsep \global\advance\bibsep by\parsep}
\newenvironment{bibsection}
    {\minipage[t]{\linewidth}\enumerate{}{%
        \setlength{\leftmargin}{\bibhang}%
        \setlength{\itemindent}{-\leftmargin}%
        \setlength{\itemsep}{\bibsep}%
        \setlength{\parsep}{\z@}%
        }}
    {\endlist\endminipage}
\makeatother

% Layout: Puts the section titles on left side of page
\reversemarginpar

%
%         PAPER SIZE, PAGE NUMBER, AND DOCUMENT LAYOUT NOTES:
%
% The next \usepackage line changes the layout for CV style section
% headings as marginal notes. It also sets up the paper size as either
% letter or A4. By default, letter was used. If A4 paper is desired,
% comment out the letterpaper lines and uncomment the a4paper lines.
%
% As you can see, the margin widths and section title widths can be
% easily adjusted.
%
% ALSO: Notice that the includefoot option can be commented OUT in order
% to put the PAGE NUMBER *IN* the bottom margin. This will make the
% effective text area larger.
%
% IF YOU WISH TO REMOVE THE ``of LASTPAGE'' next to each page number,
% see the note about the +LP and -LP lines below. Comment out the +LP
% and uncomment the -LP.
%
% IF YOU WISH TO REMOVE PAGE NUMBERS, be sure that the includefoot line
% is uncommented and ALSO uncomment the \pagestyle{empty} a few lines
% below.
%

%% Use these lines for letter-sized paper
%\usepackage[paper=letterpaper,
%            %includefoot, % Uncomment to put page number above margin
%            marginparwidth=1.2in,     % Length of section titles
%            marginparsep=.05in,       % Space between titles and text
%            margin=1in,               % 1 inch margins
%            includemp]{geometry}

%% Use these lines for A4-sized paper
\usepackage[paper=a4paper,
            %includefoot, % Uncomment to put page number above margin
            marginparwidth=28.5mm,    % Length of section titles
            marginparsep=1.5mm,       % Space between titles and text
            margin=21mm,              % 25mm margins
            includemp]{geometry}

%% More layout: Get rid of indenting throughout entire document
\setlength{\parindent}{0in}

%% This gives us fun enumeration environments. compactitem will be nice.
\usepackage{paralist}

%% Reference the last page in the page number
%
% NOTE: comment the +LP line and uncomment the -LP line to have page
%       numbers without the ``of ##'' last page reference)
%
% NOTE: uncomment the \pagestyle{empty} line to get rid of all page
%       numbers (make sure includefoot is commented out above)
%
\usepackage{fancyhdr,lastpage}
\pagestyle{fancy}
%\pagestyle{empty}      % Uncomment this to get rid of page numbers
\fancyhf{}\renewcommand{\headrulewidth}{0pt}
\fancyfootoffset{\marginparsep+\marginparwidth}
\newlength{\footpageshift}
\setlength{\footpageshift}
          {0.5\textwidth+0.5\marginparsep+0.5\marginparwidth-2in}
\lfoot{\hspace{\footpageshift}%
       \parbox{4in}{\, \hfill %
                    \arabic{page} of \protect\pageref*{LastPage} % +LP
%                    \arabic{page}                               % -LP
                    \hfill \,}}

% Finally, give us PDF bookmarks
\usepackage{color,hyperref}
\definecolor{darkblue}{rgb}{0.0,0.0,0.3}
\hypersetup{colorlinks,breaklinks,
            linkcolor=darkblue,urlcolor=darkblue,
            anchorcolor=darkblue,citecolor=darkblue}

%%%%%%%%%%%%%%%%%%%%%%%% End Document Setup %%%%%%%%%%%%%%%%%%%%%%%%%%%%


%%%%%%%%%%%%%%%%%%%%%%%%%%% Helper Commands %%%%%%%%%%%%%%%%%%%%%%%%%%%%

% The title (name) with a horizontal rule under it
%
% Usage: \makeheading{name}
%
% Place at top of document. It should be the first thing.
\newcommand{\makeheading}[1]%
        {\hspace*{-\marginparsep minus \marginparwidth}%
         \begin{minipage}[t]{\textwidth+\marginparwidth+\marginparsep}%
                {\large \bfseries #1}\\[-0.15\baselineskip]%
                 \rule{\columnwidth}{1pt}%
         \end{minipage}}

% The section headings
%
% Usage: \section{section name}
%
% Follow this section IMMEDIATELY with the first line of the section
% text. Do not put whitespace in between. That is, do this:
%
%       \section{My Information}
%       Here is my information.
%
% and NOT this:
%
%       \section{My Information}
%
%       Here is my information.
%
% Otherwise the top of the section header will not line up with the top
% of the section. Of course, using a single comment character (%) on
% empty lines allows for the function of the first example with the
% readability of the second example.
\renewcommand{\section}[2]%
        {\pagebreak[2]\vspace{1.3\baselineskip}%
         \phantomsection\addcontentsline{toc}{section}{#1}%
         \hspace{0in}%
         \marginpar{
         \raggedright \scshape #1}#2}

% An itemize-style list with lots of space between items
\newenvironment{outerlist}[1][\enskip\textbullet]%
        {\begin{itemize}[#1]}{\end{itemize}%
         \vspace{-.6\baselineskip}}

% An environment IDENTICAL to outerlist that has better pre-list spacing
% when used as the first thing in a \section
\newenvironment{lonelist}[1][\enskip\textbullet]%
        {\vspace{-\baselineskip}\begin{list}{#1}{%
        \setlength{\partopsep}{0pt}%
        \setlength{\topsep}{0pt}}}
        {\end{list}\vspace{-.6\baselineskip}}

% An itemize-style list with little space between items
\newenvironment{innerlist}[1][\enskip\textbullet]%
        {\begin{compactitem}[#1]}{\end{compactitem}}

% To add some paragraph space between lines.
% This also tells LaTeX to preferably break a page on one of these gaps
% if there is a needed pagebreak nearby.
\newcommand{\blankline}{\quad\pagebreak[2]}

%

%%%%%%%%%%%%%%%%%%%%%%%% End Helper Commands %%%%%%%%%%%%%%%%%%%%%%%%%%%

%%%%%%%%%%%%%%%%%%%%%%%%% Begin CV Document %%%%%%%%%%%%%%%%%%%%%%%%%%%%

\begin{document}
\makeheading{Niranjan Nagaraju}

%__________________________________________________________________________________________________________________
% Contact Information
\section{Contact Information}
%
% NOTE: Mind where the & separators and \\ breaks are in the following
%       table.
%
% ALSO: \rcollength is the width of the right column of the table
%       (adjust it to your liking; default is 1.85in).
%
\newlength{\rcollength}\setlength{\rcollength}{2.5in}%
%
\begin{tabular}[t]{@{}p{\textwidth-\rcollength}p{\rcollength}}
No.\ 83, H.\ Colony, & \\
2nd Cross, Indiranagar,  & \textit{Mobile:} +91 9900400100\\
Bangalore - 560038 & \textit{E-mail:} \href{mailto:niranjan.nagaraju@gmail.com}{niranjan.nagaraju@gmail.com} \\
\end{tabular}

%__________________________________________________________________________________________________________________
% Career Interests
\section{Career Interests}
%
    Algorithms Design and Optimization  \vspace{1mm}\\%
	TCP/IP Networking \vspace{1mm}\\%
	Cryptography - Algorithms and applications to network security

%__________________________________________________________________________________________________________________
% Education
\section{Education}
%
    \textbf{Visvesvaraya Technological University}, PES Institute of Technology, Bangalore \vspace{1mm}\\%
    \textsl{Bachelor of Engineering}, Computer Science \hfill \textbf{2002 -- 2006} \vspace{1mm}\\%
	Aggregate: 73.3 \% \hfill \vspace{0mm}\\\vspace{-4.5mm}%


%__________________________________________________________________________________________________________________
% Technical Skills
\section{Technical Skills} 
%	
	Programming Languages: C, C++, Python, Haskell, Linux Shell scripting (Bash) \vspace{2mm}\\%
	Tools: Oprofile, Valgrind, W3af, GDB, Vim, Emacs \vspace{2mm}\\%
	Operating Systems: Windows, Linux, Mac OS X \vspace*{2mm}\\%
	Github:  \href{https://github.com/niranjan-nagaraju/}{\textbf{Link to Github Profile}}


%__________________________________________________________________________________________________________________
% Professional Experience
\section{Professional Experience}
%    
    \textbf{5.6} years of design and development experience in System Programming, Security,\\ 
	Device driver development and TCP/IP Networking. \vspace{0mm}\\\vspace{1mm}%
	
	%__________________________________________________________________________________________________________________
	% RSA Security
    \href{http://www.rsa.com/}{\textbf{RSA Security}}, Bangalore \vspace{2mm}\\\vspace{1mm}%
    \textsl{Software Engineer} \hfill \textbf{October 2010 -- Present}%
    
    \blankline

	\textsl{RSA Access Manager (AxM) Web Agents}
	\begin{itemize}
		\item Implemented features for Intersite Single Sign-on (ISSO), Securing against arbitrary URL redirects (trusted domains).%
		\item Security audit of AxM Web agents using w3af/webscarab, Fortify and fixed critical security issues. %		
		\item Worked on extending AxM Web Agents to support Email-based RSA Adaptive Authentication. %		
		\item Responsible for fixing high-priority customer issues. \vspace{0mm}\\\vspace{1mm}%
	\end{itemize}

	%__________________________________________________________________________________________________________________
	% Narus Networks
    \href{http://www.narus.com/}{\textbf{Narus Networks}}, Bangalore \vspace{2mm}\\\vspace{1mm}%
    \textsl{Software Engineer} \hfill \textbf{June 2008 -- March 2010}%
	
	\blankline
	
	\textsl{Multi-threading processing layer}
	\begin{itemize}
		\item Worked on the development of an actor-based concurrent programming language to introduce multi-threaded efficiency in the processing layer.%
		\item Implemented Input/Output modules to be used in the new language dialects.%
		\item Profiling and performance tuning (Oprofile and Valgrind) %
	\end{itemize}

	\blankline

	\textsl{SQL Loader}
	\begin{itemize}
		\item Developed a Narus component to load capture layer metadata into an SQL database using unixODBC library. %
	\end{itemize}

	\pagebreak

	\textsl{Packet Loader}
	\begin{itemize}
		\item Developed a Narus component to analyze capture layer events/data based on user-configured targets and write the matching packets to PCAP and ETSI compliant Metadata files. \vspace{0mm}\\\vspace{1mm}%
	\end{itemize} 
 

	%__________________________________________________________________________________________________________________
	% CoreEl technologies
    \href{http://www.coreel.com/}{\textbf{CoreEl Technologies}}, Bangalore \vspace{2mm}\\\vspace{1mm}%
    \textsl{Design Engineer} \hfill \textbf{July 2006 -- June 2008}%
    
    \blankline
	
	\textsl{TOE (TCP/IPv4 Offload Engine)}
	\begin{itemize}
		\item Developed a network device driver interface to load the PCI-e based hardware stack. %
		\item Design and implementation of a dual stack architecture for the TOE stack to coexist with the Linux TCP/IP stack. %
		\item Implemented ICMP echo (Ping) and ARP handling modules in software. %
	\end{itemize}
 
 	\blankline
 
	\textsl{TIP6E (TCP/IPv6 Offload Engine)}
	\begin{itemize}
		\item Designed and implemented ICMPv6 NDP (Neighbor Discovery Protocol) and MLD (Multicast Listener Discovery) in software to be integrated with the rest of the IPv6 stack in the TIP6E hardware FPGA. %
	\end{itemize}
	
	\blankline

	\textsl{RPR (Resilient Packet Ring)}
	\begin{itemize}
		\item Implemented IOCTL interfaces for configuring the RPR stack in hardware. (RPR is a dual-ring based MAN (Metropolitan Area Network) technology). %
	\end{itemize}


%______________________________________________________________________________________________________________________
% Corporate Training
\section{Corporate Training}
%
\href{http://www.sandeepani-vlsi.com/}{\textbf{Sandeepani School of VLSI Design}}, Bangalore \hfill \textbf{December 2006 -- March 2007} \vspace{2mm}\\\vspace{1mm}%
Taught C fundamentals and Data structures courses at Sandeepani, a corporate training division of CoreEl Technologies. 


%______________________________________________________________________________________________________________________
% Internship
\section{Internship} 
%
\textbf{Hewlett Packard}, Bangalore \vspace{2mm}\\\vspace{1mm}%
    \textsl{Gelato Vanilla Project/Apache} \hfill \textbf{February 2006 -- April 2006}\\%
Benchmarking and profiling the performance of Apache server on Linux/Itanium.


%______________________________________________________________________________________________________________________
% Academic Projects
\section{Academic Projects}
%
\textsl{S.M.A.R.T. Downloader}
\begin{itemize}
	\item A download Manager written in C\#.net with zip preview, media preview, FTP support, auto-resume and concurrent multi-part download capabilities.
\end{itemize}

\blankline

\textsl{Illuminati}
\begin{itemize}
	\item An MS Paint like graphics editor program written in C/GTK+ on Linux.
\end{itemize}


%______________________________________________________________________________________________________________________
% References
\section{References} 
%
Available upon request.


%______________________________________________________________________________________________________________________
\end{document}


%______________________________________________________________________________________________________________________
% EOF
