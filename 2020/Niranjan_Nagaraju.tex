%%%%%%%%%%%%%%%%%%%%%%%%%%%%%%%%%%%%%%%%%%%%%%%%%%%%%%%%%%%%%%%%%%%%%%%%
%%%%%%%%%%%%%%%%%%%%%% Simple LaTeX CV Template %%%%%%%%%%%%%%%%%%%%%%%%
%%%%%%%%%%%%%%%%%%%%%%%%%%%%%%%%%%%%%%%%%%%%%%%%%%%%%%%%%%%%%%%%%%%%%%%%

%%%%%%%%%%%%%%%%%%%%%%%%%%%%%%%%%%%%%%%%%%%%%%%%%%%%%%%%%%%%%%%%%%%%%%%%
%% NOTE: If you find that it says                                     %%
%%                                                                    %%
%%                           1 of ??                                  %%
%%                                                                    %%
%% at the bottom of your first page, this means that the AUX file     %%
%% was not available when you ran LaTeX on this source. Simply RERUN  %%
%% LaTeX to get the ``??'' replaced with the number of the last page  %%
%% of the document. The AUX file will be generated on the first run   %%
%% of LaTeX and used on the second run to fill in all of the          %%
%% references.                                                        %%
%%%%%%%%%%%%%%%%%%%%%%%%%%%%%%%%%%%%%%%%%%%%%%%%%%%%%%%%%%%%%%%%%%%%%%%%

%%%%%%%%%%%%%%%%%%%%%%%%%%%% Document Setup %%%%%%%%%%%%%%%%%%%%%%%%%%%%

% Don't like 10pt? Try 11pt or 12pt
\documentclass[9pt]{article}

% This is a helpful package that puts math inside length specifications
\usepackage{calc}

% Simpler bibsection for CV sections
% (thanks to natbib for inspiration)
\makeatletter
\newlength{\bibhang}
\setlength{\bibhang}{1em}
\newlength{\bibsep}
 {\@listi \global\bibsep\itemsep \global\advance\bibsep by\parsep}
\newenvironment{bibsection}
    {\minipage[t]{\linewidth}\enumerate{}{%
        \setlength{\leftmargin}{\bibhang}%
        \setlength{\itemindent}{-\leftmargin}%
        \setlength{\itemsep}{\bibsep}%
        \setlength{\parsep}{\z@}%
        }}
    {\endlist\endminipage}
\makeatother

% Layout: Puts the section titles on left side of page
\reversemarginpar

%
%         PAPER SIZE, PAGE NUMBER, AND DOCUMENT LAYOUT NOTES:
%
% The next \usepackage line changes the layout for CV style section
% headings as marginal notes. It also sets up the paper size as either
% letter or A4. By default, letter was used. If A4 paper is desired,
% comment out the letterpaper lines and uncomment the a4paper lines.
%
% As you can see, the margin widths and section title widths can be
% easily adjusted.
%
% ALSO: Notice that the includefoot option can be commented OUT in order
% to put the PAGE NUMBER *IN* the bottom margin. This will make the
% effective text area larger.
%
% IF YOU WISH TO REMOVE THE ``of LASTPAGE'' next to each page number,
% see the note about the +LP and -LP lines below. Comment out the +LP
% and uncomment the -LP.
%
% IF YOU WISH TO REMOVE PAGE NUMBERS, be sure that the includefoot line
% is uncommented and ALSO uncomment the \pagestyle{empty} a few lines
% below.
%

%% Use these lines for letter-sized paper
%\usepackage[paper=letterpaper,
%            %includefoot, % Uncomment to put page number above margin
%            marginparwidth=1.2in,     % Length of section titles
%            marginparsep=.05in,       % Space between titles and text
%            margin=1in,               % 1 inch margins
%            includemp]{geometry}

%% Use these lines for A4-sized paper
\usepackage[paper=a4paper,
            %includefoot, % Uncomment to put page number above margin
            marginparwidth=28.5mm,    % Length of section titles
            marginparsep=1.5mm,       % Space between titles and text
            margin=21mm,              % 25mm margins
            includemp]{geometry}

%% More layout: Get rid of indenting throughout entire document
\setlength{\parindent}{0in}

%% This gives us fun enumeration environments. compactitem will be nice.
\usepackage{paralist}

%% Reference the last page in the page number
%
% NOTE: comment the +LP line and uncomment the -LP line to have page
%       numbers without the ``of ##'' last page reference)
%
% NOTE: uncomment the \pagestyle{empty} line to get rid of all page
%       numbers (make sure includefoot is commented out above)
%
\usepackage{fancyhdr,lastpage}
\pagestyle{fancy}
%\pagestyle{empty}      % Uncomment this to get rid of page numbers
\fancyhf{}\renewcommand{\headrulewidth}{0pt}
\fancyfootoffset{\marginparsep+\marginparwidth}
\newlength{\footpageshift}
\setlength{\footpageshift}
          {0.5\textwidth+0.5\marginparsep+0.5\marginparwidth-2in}
\lfoot{\hspace{\footpageshift}%
       \parbox{4in}{\, \hfill %
                    \arabic{page} of \protect\pageref*{LastPage} % +LP
%                    \arabic{page}                               % -LP
                    \hfill \,}}

% Finally, give us PDF bookmarks
\usepackage{color,hyperref}
\definecolor{darkblue}{rgb}{0.0,0.0,0.3}
\hypersetup{colorlinks,breaklinks,
            linkcolor=darkblue,urlcolor=darkblue,
            anchorcolor=darkblue,citecolor=darkblue}

%%%%%%%%%%%%%%%%%%%%%%%% End Document Setup %%%%%%%%%%%%%%%%%%%%%%%%%%%%


%%%%%%%%%%%%%%%%%%%%%%%%%%% Helper Commands %%%%%%%%%%%%%%%%%%%%%%%%%%%%

% The title (name) with a horizontal rule under it
%
% Usage: \makeheading{name}
%
% Place at top of document. It should be the first thing.
\newcommand{\makeheading}[1]%
        {\hspace*{-\marginparsep minus \marginparwidth}%
         \begin{minipage}[t]{\textwidth+\marginparwidth+\marginparsep}%
                {\large \bfseries #1}\\[-0.15\baselineskip]%
                 \rule{\columnwidth}{1pt}%
         \end{minipage}}

% The section headings
%
% Usage: \section{section name}
%
% Follow this section IMMEDIATELY with the first line of the section
% text. Do not put whitespace in between. That is, do this:
%
%       \section{My Information}
%       Here is my information.
%
% and NOT this:
%
%       \section{My Information}
%
%       Here is my information.
%
% Otherwise the top of the section header will not line up with the top
% of the section. Of course, using a single comment character (%) on
% empty lines allows for the function of the first example with the
% readability of the second example.
\renewcommand{\section}[2]%
        {\pagebreak[2]\vspace{1.3\baselineskip}%
         \phantomsection\addcontentsline{toc}{section}{#1}%
         \hspace{0in}%
         \marginpar{
         \raggedright \scshape #1}#2}

% An itemize-style list with lots of space between items
\newenvironment{outerlist}[1][\enskip\textbullet]%
        {\begin{itemize}[#1]}{\end{itemize}%
         \vspace{-.6\baselineskip}}

% An environment IDENTICAL to outerlist that has better pre-list spacing
% when used as the first thing in a \section
\newenvironment{lonelist}[1][\enskip\textbullet]%
        {\vspace{-\baselineskip}\begin{list}{#1}{%
        \setlength{\partopsep}{0pt}%
        \setlength{\topsep}{0pt}}}
        {\end{list}\vspace{-.6\baselineskip}}

% An itemize-style list with little space between items
\newenvironment{innerlist}[1][\enskip\textbullet]%
        {\begin{compactitem}[#1]}{\end{compactitem}}

% To add some paragraph space between lines.
% This also tells LaTeX to preferably break a page on one of these gaps
% if there is a needed pagebreak nearby.
\newcommand{\blankline}{\quad\pagebreak[2]}

%

%%%%%%%%%%%%%%%%%%%%%%%% End Helper Commands %%%%%%%%%%%%%%%%%%%%%%%%%%%

%%%%%%%%%%%%%%%%%%%%%%%%% Begin CV Document %%%%%%%%%%%%%%%%%%%%%%%%%%%%

\begin{document}
\makeheading{Niranjan Nagaraju}

%__________________________________________________________________________________________________________________
% Contact Information
\section{Contact Information}
%
% NOTE: Mind where the & separators and \\ breaks are in the following
%       table.
%
% ALSO: \rcollength is the width of the right column of the table
%       (adjust it to your liking; default is 1.85in).
%
\newlength{\rcollength}\setlength{\rcollength}{2.5in}%
%
\begin{tabular}[t]{@{}p{\textwidth-\rcollength}p{\rcollength}}
No.\ 83, H.\ Colony, & \\
2nd Main, Indiranagar,  & \textit{Mobile:} +91 9900400100\\
Bangalore - 560038 & \textit{E-mail:} \href{mailto:niranjan.nagaraju@gmail.com}{niranjan.nagaraju@gmail.com} \\
\end{tabular}


%__________________________________________________________________________________________________________________
% Education
\section{Education}
%
    \textbf{Visvesvaraya Technological University}, PES Institute of Technology, Bangalore \vspace{1mm}\\%
    \textsl{Bachelor of Engineering}, Computer Science \hfill \textbf{2002 -- 2006} \vspace{1mm}%


%__________________________________________________________________________________________________________________
% Technical Skills
\section{Technical Skills} 
%	
	Languages: C(experienced), Python(proficient), C++(familiar), Haskell(hobby) \vspace{1mm}\\%
	Tools: GDB, Vim, Emacs \vspace{1mm}\\%
	Github Profile:  \href{https://www.github.com/niranjan-nagaraju}{\textbf{www.github.com/niranjan-nagaraju}}


%__________________________________________________________________________________________________________________
% Professional Experience
\section{Professional Experience}
%    
    \textbf{13+} years of design and development experience in System Programming, Security,\\ 
	and TCP/IP Networking. \vspace{0mm}\\\vspace{1mm}%

	%__________________________________________________________________________________________________________________
	% 10Cosine
    \href{https://vividsmarthome.com/}{\textbf{10Cosine Tech Solutions Pvt Ltd}}, IoT startup, Bangalore \\%
    \textsl{Technology Consultant} \hfill \textbf{October 2018 -- Present}\\%
	
	\textsl{Vivid Smart Home, IoT-based home automation solution}
	\begin{itemize}
	\itemsep0em
		\item Optimized micropython-based firmware for an IoT-enabled smart plug to function in an ultra-low resource environment.%
		\item Desgined a network discovery mechanism for the WiFi-enabled IoT device in a local network, in the absence of traditional discovery mechanisms.\vspace{1mm}\\\vspace{1mm}%
	\end{itemize}
    
	%__________________________________________________________________________________________________________________
	% Pulse Secure
    \href{http://www.pulsesecure.net}{\textbf{Pulse Secure}}, Bangalore \\%
    \textsl{Software Engineer} \hfill \textbf{December 2017 -- August 2018}\\%
    
	\textsl{Pulse Secure Desktop Client}
	\begin{itemize}
	\itemsep0em
		\item Worked towards integrating OPSWAT host-checker with the linux desktop client.\vspace{1mm}\\\vspace{1mm}%
	\end{itemize}

	%__________________________________________________________________________________________________________________
	% Juniper Networks
    \href{http://www.juniper.net}{\textbf{Juniper Networks}}, Bangalore \\%
    \textsl{Software Engineer} \hfill \textbf{October 2012 -- November 2017}\\%
    
	\textsl{IPS (Intrusion Prevention Systems) for Juniper's SRX series}
	\begin{itemize}
	\itemsep0em
		\item Switched IPS' security database backend from Berkeley-DB to SQLite3, fine-tuned performance to match Berkeley-DB.%
        \item Integrated Intel's pattern matching engine, `Hyperscan' into SRX-IPS, reducing the memory footprint of attack signatures by 30-40\%.%
		\item Ported IPS and SSL forward-proxy modules to Intel/Linux-based next-gen SRX devices.\vspace{1mm}\\\vspace{1mm}%
	\end{itemize}

	%__________________________________________________________________________________________________________________
	% RSA Security
    \href{http://www.rsa.com/}{\textbf{RSA Security}}, Bangalore \\%
    \textsl{Software Engineer} \hfill \textbf{October 2010 -- October 2012}\\%
    
	\textsl{RSA Access Manager (AxM) Web Agents}
	\begin{itemize}
	\itemsep0em
		\item Performed a security audit of AxM Web agents using w3af/webscarab, Fortify and fixed critical security issues. %
		\item Worked on extending AxM Web Agents to support Email-based RSA Adaptive Authentication, and Intersite Single Sign-On (ISSO) features.\vspace{0mm}\\\vspace{1mm}%		
	\end{itemize}

    \pagebreak % onto page 2

	%__________________________________________________________________________________________________________________
	% Narus Networks
    \href{https://en.wikipedia.org/wiki/Narus_(company)}{\textbf{Narus Networks}}, Bangalore \\%
    \textsl{Software Engineer} \hfill \textbf{June 2008 -- March 2010}\\%
	
   \textsl{Multi-threading processing layer, `Logicserver', for NarusInsight}
	\begin{itemize}
	\itemsep0em
		\item Worked on the development of an actor-based concurrent programming language for a redesigned NarusInsight's processing layer, `Logicserver'. %
		\item Profiling and performance tuning (using Oprofile and Valgrind) of the redesigned Logicserver component. %
	\end{itemize}

	
	\textsl{SQL Loader, NarusInsight}
	\begin{itemize}
	\itemsep0em
		\item Developed a NarusInsight component to load capture layer metadata into an SQL database using unixODBC library. \vspace{0mm}\\\vspace{1mm}%
	\end{itemize}


	%__________________________________________________________________________________________________________________
	% CoreEl technologies
    \href{http://www.coreel.com/}{\textbf{CoreEl Technologies}}, Bangalore \\%
    \textsl{Design Engineer} \hfill \textbf{July 2006 -- June 2008}\\%
    
	\textsl{TOE (TCP/IPv4/IPv6 Offload Engine)}
	\begin{itemize}
	\itemsep0em
		\item Implemented a virtual network device driver for a PCI-e based FGPA device emulating TCP/IP stack in hardware. %
		\item Designed and developed a dual network-stack architecture that enables switching between TOE and the Linux TCP/IP stack. %
		\item Implemented ICMPv6 NDP (Neighbor Discovery Protocol) and MLD (Multicast Listener Discovery) protocols in software to be integrated with the IPv6 stack in the FPGA. %
	\end{itemize}
 
%______________________________________________________________________________________________________________________
% Personal Projects
\section{Personal Projects}
%
\href{https://github.com/niranjan-nagaraju/notification-center}{\textbf{Notifications Center for Android}} \hfill \textbf{2019}\\%
Developed `Notifications Center' - an Android app that organizes incoming notifications, and declutters the status bar.


%______________________________________________________________________________________________________________________
% Awards and Recognitions
\section{Awards and Recognitions}
%
\textsl{Juniper Networks} \hfill \textbf{April 2016}\\%
Received a `Department Spotlight Award' for resolving IPS accuracy issues reported during NSS Labs' Data-center tests, in a very narrow time-frame.

\blankline

\textsl{Juniper Networks} \hfill \textbf{May 2015}\\%
Our team (of three people) won 4th place at a BU-level `Innovation Day' contest for our idea titled, `Vulnerability Correlation for IPS'.
\newline The idea was to use the online CVE(Common Vulnerabilities and Exposures) database combined with network fingerprint information (OS, services running, et al.) to improve SRX-IPS attack detection.




%______________________________________________________________________________________________________________________
% References
\section{References} 
%
Available upon request.


%______________________________________________________________________________________________________________________
\end{document}


%______________________________________________________________________________________________________________________
% EOF
